%!TEX root = memoria.tex

	Agradezco a quienes contribuyeron para ir mejorando esta plantilla hecha en \LaTeX{}. Los aportes de distintas personas en el \href{http://www.industrias.usm.cl}{Departamento de Industrias}, profesores, alumnos y funcionarios fueron muy útiles para que este documento puede ser ocupado para mejorar la presentación de documentos impresos del Departamento.

	Para obtener los códigos fuente (\LaTeX{} y LyX), ir a:
	
	\url{https://bitbucket.org/jaime_rcz/usm-thesis/downloads}
	
	y descargar la versión (estable) más reciente. Para los interesados en la última versión (en desarrollo), por favor ocupar \inlinecode{git}:
	
	\inlinecode{git clone https://bitbucket.org/jaime_rcz/usm-thesis.git}

%\vspace{10mm}

\subsection*{\color{red}¡Importante!}
Este documento está preparado para ser impreso por ambos lados de una hoja (\emph{``twoside''}). Para cambiar esto, en la ``clase de documento'', reemplazar la palabra \emph{``twoside''} por \emph{``oneside''}. Es por esto que encontrará algunas hojas que están en blanco, aparentemente sin motivo.

Es posible que debas cambiar otras configuraciones también para imprimir por un sólo lado. En particular aquellas páginas en blanco después de los agradecimientos y dedicatoria.

Contribuye con el ahorro de papel, no ocupes más hojas de las necesarias.

\begin{Verbatim}[frame=lines, label=\inlinecode{/latex/memoria.tex} (extracto)
				, fontsize=\footnotesize
				, baselinestretch=1
				, formatcom=\color{gray}]
%---------------------------------------------------------------------------
%%% DOCUMENT CLASS
\documentclass[
	12pt,
	letterpaper,
	twoside
]{book}
%---------------------------------------------------------------------------
\end{Verbatim}

\newpage

\paragraph{Instrucciones para la Plantilla.}

Editar el archivo \inlinecode{/latex/agradecimientos.tex} para modificar los contenidos de esta sección.

Si no desea incluir una agradecimientos, editar el archivo \inlinecode{/latex/memoria.tex}, y comentar o borrar la sección que se muestra a continuación.

\begin{Verbatim}[frame=lines, label=\inlinecode{/latex/memoria.tex} (extracto)
, fontsize=\footnotesize
, baselinestretch=1
, formatcom=\color{gray}]
%---------------------------------------------------------------------------
%%% AGRADECMIENTOS
\thispagestyle{empty} 	% Hide Header, Footer, Page Number
\section*{Agradecimientos (Título es Opcional)}
%!TEX root = ../memoria.tex

	Agradezco a quienes contribuyeron para ir mejorando esta plantilla hecha en \LaTeX{}. 
    
    Los aportes y comentarios de distintas personas en el \href{http://www.industrias.usm.cl}{Departamento de Industrias} fueron muy útiles para que este documento puede ser ocupado para mejorar la presentación de tesis y memorias del Departamento (y la universidad).
    
    Contribuciones a este documento:
    \begin{itemize}
        \item José Miguel Gonzalez P. (\href{jose.gonzalezp@usm.cl}{jose.gonzalezp@usm.cl})
        \item Javiera Silva A. (\href{javiera.silva@usm.cl}{javiera.silva@usm.cl})
        \item Diego Cáceres S. (\href{diego.caceress@alumnos.usm.cl}{diego.caceress@alumnos.usm.cl})
    \end{itemize}
    \hfill Gracias!!!


\vspace{20mm}
\begin{framed}
\noindent\textbf{\color{red}Para el impaciente ...}

Descargar plantilla desde : \url{https://jaimercz.github.io/utfsm-thesis}
	
    
\noindent O, si busca la última versión, por favor ocupar \inlinecode{git}:

	\inlinecode{git clone https://github.com/jaimercz/utfsm-thesis.git}

Para los interesados en \inlinecode{git} revisar \citet{git2017}.


\end{framed}	% Archivo agradecimientos.tex
\newpage\thispagestyle{empty}\cleardoublepage
%---------------------------------------------------------------------------
\end{Verbatim}

\paragraph{Codificación de caracteres.}

Todos los archivos \inlinecode{*.tex} de esta plantilla han sido preparados ocupando la codificación de caracteres por defecto \emph{unicode} (UTF-8). Windows (y algunas versiones de OSX) ocupan otro tipo de codificación (ej. \emph{Windows-1252} o \emph{Mac Roman}) por defecto. Si deseas ocupar esta plantilla y encuentras problemas con los caracteres acentuados, entonces puedes optar por una de estas tres alternativas:
\begin{enumerate}[i)]
    \item cambiar tu editor (TexMaker, TexStudio, TexShop, etc.) para que ocupe UTF-8 como codificación de caracteres por defecto; o
    \item cambiar la codificación de cada documento \inlinecode{*.tex} para que ocupe la codificación nativa de tu sistema operativo; y, sustituir en el archivo \inlinecode{memoria.tex} la línea (\#62) que dice:
    
    \inlinecode{\\usepackage[utf8x]\{inputenc\}}, por el texto \inlinecode{\\usepackage[latin1]\{inputenc\}}.
    \item escribir todo ocupando caracteres pre-acentuados (ej. \inlinecode{\\'a} en lugar de á).
\end{enumerate}

\begin{framed}
    \textbf{Recuerda:} Mezclar documentos de distintas codificaciones puede generarte muchos problemas al momento de compilar.  
\end{framed}
